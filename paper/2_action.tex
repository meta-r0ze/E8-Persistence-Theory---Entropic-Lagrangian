\section{Theoretical Framework: The Entropic Action}
\label{sec:theory_framework}

Before deriving the specific operators of the Standard Model, we must define the quantity that the universe is optimizing. Standard physics assumes the Principle of Least Action ($\delta S = 0$) as an axiom. We derive it as a requirement for persistence.

\subsection{The Thermodynamic Dual}
Open systems persist by maximizing energy intake; however, the vacuum is a closed system. Therefore, the persistence criterion shifts: rather than maximizing intake, the vacuum must \textbf{minimize information loss}.

We define the \textbf{Dissipation Functional} ($\Phi_{drag}$) as the rate of irreversible information loss. According to Landauer's Principle, processing information requires energy, and erasing information generates heat (entropy). 

\begin{equation}
\Phi_{drag}[\psi] = \xi \cdot \mathbf{K}[\psi] \cdot \lambda_{flux}[\psi]
\end{equation}

Where:
\begin{itemize}
    \item $\xi$ (\textbf{Landauer Coefficient}): The energetic cost of a bit transition, $\xi = k_B T_{eff} \ln 2$. Here, $T_{eff} \approx m_e/k_B$ defines the thermal scale of the vacuum noise floor.
    \item $\mathbf{K}[\psi]$ (\textbf{Structural Complexity}): The information density of the state (Bits). For a particle, this corresponds to Mass (Capacity).
    \item $\lambda_{flux}[\psi]$ (\textbf{Transition Flux}): The rate of state erasure or decoherence (Hz). This corresponds to the Kinetic energy (Protocol).
\end{itemize}

\subsection{The Entropic Action Integral}
To maintain a consistent history, we extend the instantaneous Entropic Density to a cosmic integral defining the \textbf{Entropic Action} ($S_\Phi$). The Persistence Principle is formally stated as the requirement that the vacuum geometry be a stationary point of this action:

\begin{equation}
S_\Phi = \int d^4x \, \mathcal{L}_\Phi = \int d^4x \, \xi \cdot \mathbf{K}[\psi] \cdot \lambda_{flux}[\psi]
\end{equation}

In the continuum limit, the minimization of $\Phi_{\text{drag}}$ at fixed particle number (conserved $\int \mathbf{K} \, d^4x$) reduces to minimizing the gradient energy, yielding the canonical kinetic term.

\subsection{Mapping Rules: Information to Fields}
To recover the Standard Model, we map these information-theoretic operators to continuous field operators using two general rules:

\begin{enumerate}
    \item \textbf{Structure $\rightarrow$ Quadratic Invariant:} For probability to be conserved (Unitarity), the information density $\mathbf{K}$ must map to the norm of the field: 
    $$ \mathbf{K}[\Psi] \propto \Psi^\dagger \Psi $$
    \item \textbf{Flux $\rightarrow$ Covariant Derivative:} To satisfy the Protocol Constraint (Local Gauge Invariance), the rate of change $\lambda_{flux}$ must promote ordinary gradients to covariant derivatives: 
    $$ \lambda_{flux}[\Psi] \propto (D_\mu \Psi)^\dagger (D^\mu \Psi) $$
\end{enumerate}

We now apply these rules to the specific geometry of the lattice to derive the Standard Model Lagrangian.

\section{System III: The Entropic Dynamics (The Lagrangian)}

Having defined the substrate geometry, we now derive the equations of motion. To do so, we must first establish the metric constraints imposed by the discrete nature of the lattice.

\subsection{The Kinetic Constraint: Lorentz Invariance as Channel Capacity}
\label{sec:kinetic_constraint}

In the companion paper \textit{Informational Energetics}, we established that the substrate is a discrete lattice with a fixed update frequency $\Delta$ and node spacing $\ell$. This creates a hard physical limit on information propagation:
\begin{equation}
    c_{eff} \equiv \frac{\ell}{\Delta} = 1
\end{equation}
In standard physics, Lorentz Invariance is treated as a geometric axiom. In the Entropic Lagrangian framework, we identify it as the \textbf{Shannon Channel Capacity} ($C_{max}$) of the vacuum. The speed of light is the maximum rate of causality; information can propagate exactly one node spacing per update cycle.

For the Lagrangian $\mathcal{L}_{\Phi}$ to describe a persistent system, it must enforce this capacity limit dynamically. It does so by assigning an infinite Entropic Action cost to any trajectory that attempts to exceed $c_{eff}$. This requirement necessitates the use of the \textbf{Minkowski Metric} ($\eta_{\mu\nu} = \text{diag}(-1, 1, 1, 1)$) in the kinetic terms.

Specifically, the geometric distance $ds^2 = -dt^2 + dx^2$ acts as a \textbf{Causality Filter} (The Governor):
\begin{itemize}
    \item \textbf{Timelike ($ds^2 < 0$):} The signal is within bandwidth ($v < c$). The action is real and finite. Persistence is possible.
    \item \textbf{Lightlike ($ds^2 = 0$):} The signal is at the bandwidth limit ($v = c$). This defines the propagation of massless carriers (The Protocol).
    \item \textbf{Spacelike ($ds^2 > 0$):} The signal exceeds bandwidth ($v > c$). This results in causal aliasing.
\end{itemize}

Therefore, the kinetic terms of the Standard Model are not merely defining motion; they are the \textbf{Penalty Functions} applied by the system to suppress causal violations. We now apply this metric constraint to derive the specific operators for Matter, Forces, and Geometry.