\section{Derivation A: The Dirac Propagator from Flux Conservation (Map)}
Time is treated as a discrete clock cycle defined by the fundamental resonance $\Delta$. The state of a node $\psi(x, t)$ updates based on the flux from its geometric neighbors.

\subsection{Unitary Flux Constraint}

The Unitary Flux Constraint is based on the conservation of information flux, analogous to Kirchhoff's Current Law in network theory. This approach is formally equivalent to a Quantum Lattice Gas Automaton (QLGA)\cite{meyer_quantum_1996} or the discrete Feynman Checkerboard model \cite{feynman_quantum_1965}, extended here to the $E_8$ geometry.

The update is the sum of inflows from the chiral lattice vectors $e_k$. Define the discrete evolution as:
\begin{equation}
\begin{split}
    \psi(x, t+\delta t) =& \psi(x, t) \\
    & - i \mathcal{A} \sum_{k=1}^{3} \left( \psi(x + e_k \delta x, t) - \psi(x - e_k \delta x, t) \right) \\
    & - i m \psi(x, t) \\
\end{split}
\label{eq:KirchhoffUpdateRule}
\end{equation}
Where $\mathcal{A}$ is the vacuum admittance (coupling strength) and $m$ is the mass impedance.

\subsection{The Continuum Limit}
Rearranging the update rule to form difference quotients:
\begin{equation}
\begin{split}
    \frac{\psi(x, t+\delta t) - \psi(x,t)}{\delta t} = \\
    -i \mathcal{A} \sum_{k} \frac{\psi(x + e_k \delta x, t) - \psi(x - e_k \delta x, t)}{\delta x} \frac{\delta x}{\delta t} 
    - i m \psi(x, t)
\end{split}
\end{equation}

In the continuum limit ($\delta t, \delta x \to 0$), keeping $c = \delta x / \delta t = 1$, the left side becomes $\partial_t \psi$ and then Taylor expand the right side. Identify the vacuum admittance with the unit coupling ($\mathcal{A} = 1$ in natural units), and separate the temporal and spatial components:

\begin{equation}
    i\partial_t \psi = i\sum_{k=1}^{3} \gamma^k \partial_k \psi - m\psi
\end{equation}

\subsection{Dirac equation}
Identifying the temporal update vector with the time-like gamma matrix ($\gamma^0$), we obtain the covariant form of the Dirac equation:
\begin{equation}
    (i\gamma^\mu \partial_\mu - m)\psi = 0
\end{equation}

\subsection{The Clifford Identification}
The jump from lattice vectors $e_k$ to Dirac matrices $\gamma^\mu$ requires justification. The key geometric insight is that the $D_4$ lattice basis vectors are not ordinary spatial directions, but \textbf{spinor generators}. Spinor representation of $D_4$ requires a 4-dimensional basis satisfying the metric of the projection manifold:

\begin{equation}
    \{e_\mu, e_\nu\} \equiv e_\mu e_\nu + e_\nu e_\mu = 2g_{\mu\nu}
\end{equation}

where $g_{\mu\nu} = \text{diag}(-1,+1,+1,+1)$ is the Lorentzian metric for the $E_8$ substrate. This anti-commutation relation is precisely the defining property of the Dirac gamma matrices. Thus, the lattice geometry does not merely \textit{map} to the Dirac equation; it structurally \textit{is} the Clifford algebra.