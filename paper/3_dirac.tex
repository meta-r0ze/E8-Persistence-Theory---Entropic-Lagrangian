\section{Derivation A: The Dirac Propagator (The Map)}
\label{sec:dirac_derivation}

We derive the dynamics of the \textbf{Map} pillar by defining the evolution of a state $\psi(x,t)$ on the discrete lattice. Time is treated not as a continuous dimension, but as a discrete clock cycle defined by the fundamental resonance $\Delta$.

\subsection{The Unitary Flux Constraint}
The evolution of the state is governed by the conservation of probability amplitude (Unitarity). We define the update rule based on the \textbf{Quantum Lattice Gas Automaton (QLGA)} formalism \cite{meyer_quantum_1996}, which generalizes the Feynman Checkerboard model \cite{feynman_quantum_1965} to the $D_4$ geometry.

The state at a node $\psi(x, t+\delta t)$ is the superposition of its previous state and the information flux received from its chiral neighbors. This is the \textbf{Kirchhoff Condition for Information}: the net flux into a node determines the rate of change of the state amplitude.

We define the discrete evolution operator as:
\begin{equation}
\begin{split}
    \psi(x, t+\delta t) =& \psi(x, t) \\
    & - i \mathcal{A} \sum_{k=1}^{3} \left[ \psi(x + e_k \delta x, t) - \psi(x - e_k \delta x, t) \right] \\
    & - i m \psi(x, t)
\end{split}
\label{eq:LatticeUpdateRule}
\end{equation}
Where:
\begin{itemize}
    \item $e_k$ are the lattice basis vectors.
    \item $\mathcal{A}$ is the vacuum admittance (coupling efficiency).
    \item $m$ is the topological impedance (mass), representing the cost of state retention.
    \item The factor $i$ ensures unitary rotation rather than diffusive decay.
\end{itemize}

\subsection{The Continuum Limit}
To recover the field equation, we rearrange the update rule into difference quotients. We take the limit where the lattice spacing $\delta x$ and time step $\delta t$ approach zero, while fixing the ratio $c \equiv \delta x / \delta t = 1$ (The Bandwidth Limit):

\begin{equation}
\begin{split}
    \frac{\psi(t+\delta t) - \psi(t)}{\delta t} = \\ 
    -i \mathcal{A} \sum_{k} \left( \frac{\psi(x + e_k \delta x) - \psi(x - e_k \delta x)}{\delta x} \right) \frac{\delta x}{\delta t} 
    - i m \psi(t)
    \end{split}
\end{equation}

Recognizing the central difference quotient as the spatial derivative $\partial_k$ and the left-hand side as the temporal derivative $\partial_t$:

\begin{equation}
    \partial_t \psi = -i \sum_{k=1}^{3} \Gamma^k \partial_k \psi - i m \psi
\end{equation}

Multiplying by $i$ to match standard quantum mechanical notation:
\begin{equation}
    i \partial_t \psi = \sum_{k=1}^{3} \Gamma^k \partial_k \psi + m \psi
\end{equation}
Here, $\Gamma^k$ represents the projection of the lattice vectors onto the propagation manifold.

\subsection{The Clifford Identification}
The critical step in identifying this lattice rule with the Dirac equation is the geometric nature of the basis vectors $e_k$. In the $E_8 \to D_4$ projection, the lattice basis vectors are not simple coordinate directions; they are \textbf{Spinor Generators}.

For the propagator to be consistent on the projection manifold, the basis vectors must satisfy the metric of that manifold. The anti-commutation relation for the $D_4$ lattice basis is:
\begin{equation}
    \{e_\mu, e_\nu\} \equiv e_\mu e_\nu + e_\nu e_\mu = 2g_{\mu\nu} \mathbb{I}
\end{equation}
where $g_{\mu\nu} = \text{diag}(-1,+1,+1,+1)$ is the Lorentzian metric signature derived from the causal projection.

This relation is precisely the defining algebra of the Dirac gamma matrices ($\gamma^\mu$). Therefore, we identify the lattice update vectors $e_\mu$ directly with the gamma matrices:
\begin{equation}
    (e_0, e_1, e_2, e_3) \equiv (\gamma^0, \gamma^1, \gamma^2, \gamma^3)
\end{equation}

Substituting this identification into the continuum equation and multiplying by $\gamma^0$ (the temporal vector) yields the covariant Dirac equation:
\begin{equation}
    (i\gamma^\mu \partial_\mu - m)\psi = 0
\end{equation}

\subsection{The Chiral Filter: Geometric Parity Violation}
Standard QFT introduces the chiral projection operator $P_L = \frac{1-\gamma^5}{2}$ as an empirical constraint to match the observed weak interaction. In the E8-Persistence framework, this is a geometric necessity of the projection.

The decomposition $E_8 \to D_4 \oplus D_4$ splits the 16-component lattice state into two orthogonal sectors. Crucially, the causal metric signature $(-1, +1, +1, +1)$ required for the \textbf{Toll} pillar (Time) can only be assigned to one $D_4$ sector (Spacetime). The orthogonal $D_4$ sector (Internal Symmetry) remains Euclidean $(+1, +1, +1, +1)$.

Dynamics requires a time derivative ($\partial_t$). States residing in the Euclidean sector possess no temporal generator; they are geometrically static ($\partial_t \psi_{mirror} \equiv 0$) relative to the causal manifold. Consequently, the "Right-Handed" (Mirror) components of the $E_8$ spinor are structurally filtered out of the propagation dynamics.
The Dirac propagator on the projected manifold acts only on the active sub-algebra, enforcing maximum parity violation ($P_L$) not by suppression, but by the topological absence of a temporal update path for the mirror sector.

\subsection{The Capacity Mechanism: Mass as Chirality Mixing}
The massless Dirac equation describes pure ballistic transport at $c=1$ ($\mathcal{L} \propto \bar{\psi} \slashed{D} \psi$). This satisfies the \textbf{Map} and \textbf{Protocol} pillars (propagation) but violates the \textbf{Capacity} pillar: a signal moving at $c$ stores zero information locally (no rest frame).

To satisfy Capacity, the node must retain state over time. In the lattice formalism, this requires converting an outgoing flux vector (Left-Chiral) into an incoming flux vector (Right-Chiral), creating a local resonance or ``Zitterbewegung'' loop.
Mathematically, this requires a coupling term that mixes the orthogonal chiral sectors:
\begin{equation}
    \mathcal{L}_{Cap} \propto \psi_L^\dagger \psi_R + \psi_R^\dagger \psi_L = \bar{\psi} \psi
\end{equation}
This derivation proves that the scalar mass term $-m\bar{\psi}\psi$ is the unique algebraic form capable of arresting information flow. It is the "Reflection Coefficient" of the lattice node, defining the energetic cost ($m$) of preventing the state from dissipating at the speed of light.

\textbf{Conclusion:} The Dirac equation is not an axiom. It is the hydrodynamic limit of a discrete, unitary cellular automaton evolving on a spinor lattice. The "matrices" of the standard model are simply the basis vectors of the underlying geometry.